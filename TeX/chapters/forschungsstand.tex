% TODO: Durchlesen!

\section{Forschungsstand}

Die Auswirkungen von Digitalisierung und \ac{ICT}-Investitionen auf den Arbeitsmarkt sind 
ein zentrales Thema der arbeitsmarktökonomischen Forschung. Während einige Studien den 
Fokus auf die technologische Verdrängung bestimmter Berufsgruppen legen, untersuchen 
andere, inwieweit institutionelle Rahmenbedingungen wie Wohlfahrtsstaaten die Effekte von 
Digitalisierung abmildern oder verstärken. In diesem Kapitel werden zunächst die 
allgemeinen Auswirkungen der Digitalisierung auf Arbeitsmärkte analysiert, bevor der Fokus 
auf die Rolle von \ac{ICT}-Investitionen und die Unterschiede zwischen verschiedenen 
Wohlfahrtsstaaten gelegt wird. Abschließend werden bestehende Forschungslücken aufgezeigt, 
die eine weiterführende Analyse notwendig machen.

%%%%%%%%%%%%%%%%%%%%%%%%%%%%%%%%%%%%%%%%%%%%%%%%%%%%%%
% Auswirkungen der Digitalisierung auf Arbeitsmärkte %
%%%%%%%%%%%%%%%%%%%%%%%%%%%%%%%%%%%%%%%%%%%%%%%%%%%%%%

\subsection{Auswirkungen der Digitalisierung auf Arbeitsmärkte}

Die Digitalisierung und insbesondere Investitionen in \ac{ICT} haben die Arbeitsmärkte 
weltweit in den letzten Jahrzehnten grundlegend verändert. Empirische Studien zeigen, dass 
diese Entwicklungen die Beschäftigungsstrukturen in verschiedenen Bildungsgruppen 
unterschiedlich beeinflussen. Die Automatisierung technologischer Prozesse sowie der 
verstärkte Einsatz digitaler Systeme betreffen insbesondere Tätigkeiten, die routinisierbar 
und standardisierbar sind. In diesem Zusammenhang haben verschiedene wissenschaftliche 
Arbeiten untersucht, welche Beschäftigungsgruppen besonders von den digitalen 
Transformationsprozessen betroffen sind.

Eine der einflussreichsten Arbeiten in diesem Bereich stammt von Autor, Levy und Murnane 
(2003), die zeigen, dass Tätigkeiten mit einem hohen Anteil an routinemäßigen kognitiven 
und manuellen Aufgaben besonders anfällig für Automatisierung sind. Die daraus resultierende 
Theorie des \ac{RBTC} besagt, dass Digitalisierung insbesondere Arbeitsplätze betrifft, die 
in klar definierten, sich wiederholenden Mustern ablaufen und daher relativ leicht durch 
Algorithmen und Maschinen ersetzt werden können \parencite[S. 1281]{autor2003theskill}. 
Frey und Osborne (2017) erweiterten diese Analyse und quantifizierten das 
Substitutionsrisiko durch Automatisierung für verschiedene Berufsgruppen. Ihre 
Untersuchung für den US-amerikanischen Arbeitsmarkt ergab, dass bis zu 47\% der 
Arbeitsplätze potenziell durch Automatisierung ersetzt werden könnten 
\parencite[S. 254]{frey2013thefuture}. Dabei stellten sie fest, dass vor allem Berufe mit 
niedrigem Qualifikationsniveau gefährdet sind, insbesondere in den Bereichen administrative 
Tätigkeiten, einfache Produktionstätigkeiten und Transportwesen. Diese Ergebnisse sind auch 
auf viele \ac{OECD}-Länder übertragbar, da ähnliche Beschäftigungsstrukturen und 
Automatisierungstrends in entwickelten Volkswirtschaften zu beobachten sind.

Parallel zur Automatisierung zeigt sich eine Polarisierung der Arbeitsmärkte, die von Goos, 
Manning und Salomons (2014) ausführlich analysiert wurde. Ihre Untersuchung belegt, dass die 
Digitalisierung gleichzeitig Berufe stärkt, die komplexe kognitive, kreative oder soziale 
Fähigkeiten erfordern \parencite[S. 2509]{goos2014explaining}. Während mittlere 
Qualifikationsgruppen unter Druck geraten, profitieren insbesondere hochqualifizierte 
Beschäftigte, die über spezialisierte technologische Kenntnisse verfügen, von der 
steigenden Nachfrage nach digitalen und analytischen Fähigkeiten. Diese Entwicklung führt 
dazu, dass gut ausgebildete Arbeitskräfte mit hohen Qualifikationen von der 
Digitalisierung profitieren, während gering Qualifizierte in wachsendem Maße von 
Arbeitsplatzverlusten betroffen sind. Dies verstärkt das Risiko sozialer Ungleichheit, da 
Beschäftigungschancen zunehmend ungleich verteilt sind. Diese Divergenz wird häufig als 
„Digital Divide“ bezeichnet, da sie die Kluft zwischen hoch- und niedrigqualifizierten 
Arbeitskräften weiter vertieft \parencite[S. 10]{acemoglu2002technical}.

Die Konsequenzen der Digitalisierung variieren zudem stark nach Branche und 
Wirtschaftssektor. Während sich einige Sektoren wie die Industrieproduktion oder der 
Einzelhandel durch die Einführung automatisierter Systeme massiv verändert haben, 
profitieren andere Sektoren, wie die wissensintensive Dienstleistungsbranche oder das 
Gesundheitswesen, von den neuen technologischen Möglichkeiten 
\parencite[S. 1555]{autor2013thegrowth}. Besonders betroffen sind manuelle Tätigkeiten in 
der Fertigungsindustrie sowie administrative Büroarbeiten, die zunehmend durch 
algorithmische Prozesse ersetzt werden \parencite[S. 260]{frey2013thefuture}. 
Gleichzeitig entstehen jedoch auch neue Arbeitsplätze, insbesondere in den Bereichen 
Datenwissenschaft, IT-Entwicklung, Robotik und künstliche Intelligenz 
\parencite[S. 2510]{goos2014explaining}. In diesen Berufsfeldern wächst die Nachfrage nach 
spezialisierten Fachkräften, sodass die Digitalisierung nicht nur Verdrängungseffekte auf 
dem Arbeitsmarkt mit sich bringt, sondern auch neue Qualifikationsanforderungen schafft.

Die Digitalisierung verändert Arbeitsmärkte auf mehreren Ebenen: Einerseits verstärkt sie 
das Risiko der Automatisierung insbesondere für Berufe mit mittlerem und niedrigem 
Qualifikationsniveau, andererseits eröffnet sie neue Beschäftigungsmöglichkeiten für 
Hochqualifizierte \parencite[S. 1555]{autor2013thegrowth}. Die zunehmende Kluft zwischen 
verschiedenen Qualifikationsgruppen hat tiefgreifende Auswirkungen auf die 
Einkommensverteilung, soziale Mobilität und die Notwendigkeit gezielter 
arbeitsmarktpolitischer Maßnahmen \parencite[S. 2510]{goos2014explaining}. In der 
vorliegenden Arbeit wird untersucht, inwieweit \ac{ICT}-Investitionen in OECD-Ländern mit 
der Arbeitslosigkeit in verschiedenen Bildungsgruppen korrelieren und ob sich Muster 
zwischen unterschiedlichen Wohlfahrtsstaaten identifizieren lassen.

%%%%%%%%%%%%%%%%%%%%%%%%%%%%%%%%%%%%%%%%%%%%%%%%%%%%
% ICT-Investitionen als Treiber der Transformation %
%%%%%%%%%%%%%%%%%%%%%%%%%%%%%%%%%%%%%%%%%%%%%%%%%%%%

\subsection{ICT-Investitionen als Treiber der Transformation}

Investitionen in \ac{ICT} gelten als zentraler Indikator für den 
Digitalisierungsgrad eines Landes und spielen eine Schlüsselrolle bei der Transformation 
moderner Arbeitsmärkte. Der verstärkte Einsatz digitaler Technologien verändert 
Produktions- und Geschäftsprozesse grundlegend und beeinflusst die Nachfrage nach 
Arbeitskräften in verschiedenen Qualifikationsgruppen. Empirische Studien zeigen, dass 
Unternehmen, die verstärkt in \ac{ICT} investieren, effizientere Abläufe entwickeln, ihre 
Wettbewerbsfähigkeit steigern und tendenziell eine höhere Nachfrage nach qualifizierten 
Arbeitskräften verzeichnen \parencite[S. 12]{corrado2018intangible}.

Der Einfluss von \ac{ICT}-Investitionen auf den Arbeitsmarkt ist dabei vielschichtig. Laut 
der \ac{OECD} (2019) ermöglichen diese Investitionen nicht nur eine zunehmende 
Automatisierung, sondern tragen auch zur Integration globaler Wertschöpfungsketten bei 
und treiben das wirtschaftliche Wachstum voran \parencite[S. 15ff]{oecd2019measuring}. 
Der Wandel vollzieht sich insbesondere im hohen Bildungssektor, in dem Dienstleistungen 
zunehmend digitalisiert werden und damit steigende Qualifikationsanforderungen entstehen. 
Dies betrifft insbesondere Bereiche wie Finanzdienstleistungen, IT-gestützte 
Geschäftsprozesse, E-Commerce oder digitale Plattformarbeit, bei denen neue 
Geschäftsmodelle entstehen, die verstärkt auf Automatisierung und datenbasierte 
Entscheidungsprozesse setzen.

Während \ac{ICT}-Investitionen also zur Schaffung neuer Arbeitsplätze führen können, zeigen 
zahlreiche Studien, dass diese Transformation auch polarisierende Effekte mit sich bringt. 
Hochqualifizierte Arbeitskräfte profitieren von der steigenden Nachfrage nach digitalen und 
analytischen Fähigkeiten, während geringqualifizierte Beschäftigte einem zunehmenden Risiko 
der Arbeitsplatzverdrängung ausgesetzt sind \parencite[Kap. 2]{brynjolfsson2015thesecond}. 
Besonders betroffen sind Tätigkeiten mit einem hohen Anteil an repetitiven, 
standardisierten Prozessen, die sich leicht durch digitale Technologien oder künstliche 
Intelligenz ersetzen lassen. Dazu zählen nicht nur manuelle Produktionsprozesse, sondern 
auch administrative Tätigkeiten im Bürobereich, die zunehmend durch automatisierte 
Softwarelösungen abgelöst werden.

Die sich verstärkende Polarisierung des Arbeitsmarktes ist eng mit der Theorie des \ac{SBTC}
verbunden, die davon ausgeht, dass technologischer Fortschritt die Nachfrage nach 
hochqualifizierten Arbeitskräften erhöht, während Tätigkeiten mit mittlerem 
Qualifikationsniveau unter Druck geraten \parencite[S. 22]{acemoglu2002technical}. Diese 
Entwicklung führt zu einer Verschiebung in der Beschäftigungsstruktur, da insbesondere 
wissensintensive Berufe von \ac{ICT}-Investitionen profitieren, während traditionelle 
Berufe in der industriellen Fertigung oder im einfachen Dienstleistungsbereich zunehmend 
verdrängt werden.

Gleichzeitig zeigt sich, dass \ac{ICT}-Investitionen nicht in allen Ländern und Branchen 
gleichermaßen produktivitätssteigernd wirken. Ihre Effekte hängen stark von begleitenden 
wirtschaftspolitischen Maßnahmen ab, darunter Investitionen in digitale Infrastruktur, die 
Förderung digitaler Kompetenzen und die Anpassung von Bildungsprogrammen an die veränderten 
Anforderungen des Arbeitsmarktes \parencite[S. 77]{brynjolfsson2015thesecond}. Länder mit 
einer gezielten digitalen Transformationsstrategie, wie etwa Südkorea oder die 
skandinavischen Staaten, konnten in den letzten Jahrzehnten eine positive Korrelation 
zwischen \ac{ICT}-Investitionen und Wirtschaftswachstum feststellen 
\parencite[S. 34]{oecd2020digital}. Empirische Studien zeigen, dass digitale Infrastruktur 
und eine strategische Förderung von digitaler Bildung eine zentrale Rolle dabei spielen, 
die wirtschaftlichen Vorteile von ICT-Investitionen vollständig auszuschöpfen 
\parencite[S. 360]{vu2011ict}. Länder mit einem schwächeren Fokus auf digitale Bildung 
haben größere Schwierigkeiten, von diesen Entwicklungen zu profitieren, da der Mangel an 
digitalen Kompetenzen die Innovationskraft und Produktivität hemmt 
\parencite[S. 34]{oecd2020digital}.

Zusammenfassend zeigen \ac{ICT}-Investitionen sowohl wachstumsfördernde als auch 
polarisierende Effekte auf den Arbeitsmarkt. Während sie die Produktivität und 
Wettbewerbsfähigkeit von Unternehmen steigern und neue Beschäftigungsmöglichkeiten für 
hochqualifizierte Arbeitskräfte schaffen, verstärken sie gleichzeitig das Risiko der 
Arbeitsplatzverdrängung für geringqualifizierte Arbeitskräfte. Die Digitalisierung des 
Dienstleistungssektors, die zunehmende Automatisierung administrativer Prozesse und die 
Integration neuer Technologien in industrielle Produktionsabläufe führen dazu, dass 
traditionelle Berufsbilder zunehmend hinterfragt und an neue Anforderungen angepasst werden 
müssen. Diese Entwicklungen unterstreichen die Notwendigkeit arbeitsmarktpolitischer 
Maßnahmen, um den Wandel sozial abzufedern und die Vorteile der Digitalisierung möglichst 
breit in der Gesellschaft zu verteilen.

%%%%%%%%%%%%%%%%%%%%%%%%%%%%%%%%%%%%%%%%%%%
% Unterschiede zwischen Wohlfahrtsstaaten %
%%%%%%%%%%%%%%%%%%%%%%%%%%%%%%%%%%%%%%%%%%%

\subsection{Unterschiede zwischen Wohlfahrtsstaaten}

Empirische Studien zeigen, dass die Auswirkungen von Digitalisierung und 
\ac{ICT}-Investitionen auf Arbeitsmärkte stark von den institutionellen Rahmenbedingungen 
eines Landes abhängen. Regierungen spielen eine zentrale Rolle bei der Förderung digitaler 
Infrastruktur, der Implementierung von Bildungspolitik und der Regulierung des 
Arbeitsmarktes \parencite{hall2001varieties}. Studien haben gezeigt, dass Länder mit hohen 
Investitionen in digitale Bildung und Infrastruktur tendenziell bessere Anpassungsprozesse 
an den technologischen Wandel durchlaufen \parencite{oecd2020digital}.

Laut der OECD (2019) variieren die Investitionen in Informations- und 
Kommunikationstechnologie erheblich zwischen Ländern. Skandinavische Staaten und die 
Niederlande investieren überdurchschnittlich in digitale Bildung und Innovationen, während 
süd- und osteuropäische Länder vergleichsweise niedrigere Investitionen tätigen 
\parencite[S. 45]{oecd2020digital}.

Länder mit hoher ICT-Investitionsquote (z. B. Schweden, Niederlande) zeigen niedrigere 
Arbeitslosenquoten unter Hochqualifizierten und profitieren von einer stärkeren Nachfrage 
nach digitalen Kompetenzen \parencite[S. 78]{brynjolfsson2015thesecond}.
Länder mit geringeren ICT-Investitionen (z. B. Spanien, Ungarn) sind stärker von 
Automatisierung betroffen, da ein größerer Anteil der Beschäftigten in Routineberufen mit 
mittlerem Qualifikationsniveau tätig ist \parencite[S. 12]{frey2013thefuture}.

Neben Investitionen in Digitalisierung spielen staatliche Bildungs- und 
Arbeitsmarktpolitiken eine entscheidende Rolle für die Fähigkeit eines Landes, sich an 
technologische Veränderungen anzupassen. Länder mit umfassenden Umschulungs- und 
Weiterbildungsprogrammen (z. B. Dänemark, Deutschland) haben bessere Voraussetzungen, um 
durch lebenslanges Lernen den digitalen Wandel sozialverträglich zu gestalten 
\parencite[S. 361]{vu2011ict}. Staaten mit weniger regulierten Arbeitsmärkten (z. B. 
USA, Großbritannien) haben eine schnellere, aber oft ungleichere Anpassung an 
technologische Innovationen, was zu verstärkter Arbeitsplatzpolarisierung führen kann 
\parencite[S. 172]{goos2014explaining}.

Studien zeigen, dass das Automatisierungsrisiko je nach Land und Wirtschaftsstruktur stark 
variiert. Laut einer OECD-Analyse von Arntz, Gregory \& Zierahn (2016) sind in süd- und 
osteuropäischen Ländern bis zu 40\% der Arbeitsplätze einem hohen Automatisierungsrisiko 
ausgesetzt, während es in skandinavischen Ländern und Deutschland nur etwa 20–25\% sind 
\parencite[S. 12]{arntz2016therisk}.

Ein entscheidender Faktor für diese Unterschiede ist die Wirtschaftsstruktur: Länder mit 
einem hohen Anteil wissensintensiver Dienstleistungen (z. B. Schweden, Niederlande) sind 
weniger von Automatisierung betroffen. Industrie- und produktionslastige Länder (z. B. 
Spanien, Polen) weisen höhere Risiken für Arbeitsplatzverluste durch Automatisierung auf 
\parencite[S. 260]{frey2013thefuture}.

%%%%%%%%%%%%%%%%%%%%
% Forschungslücken %
%%%%%%%%%%%%%%%%%%%%

\subsection{Forschungslücken}

Obwohl zahlreiche Studien die Auswirkungen von Digitalisierung und \ac{ICT}-Investitionen 
untersuchen, bestehen weiterhin relevante Forschungslücken. Die Mehrheit der bisherigen 
Studien konzentriert sich auf die allgemeinen Effekte von Digitalisierung auf den 
Arbeitsmarkt, ohne spezifisch zwischen verschiedenen Wohlfahrtsstaatentypen zu 
unterscheiden. Es fehlen systematische Vergleiche, die institutionelle Faktoren wie 
Bildungssysteme und Arbeitsmarktregulierungen einbeziehen. Viele empirische Studien zur 
Automatisierung betrachten vorwiegend die Situation in den USA, wohingegen umfassende 
Analysen für \ac{OECD}-Länder mit unterschiedlichen Wohlfahrtsmodellen begrenzt sind.

Der langfristige Einfluss von \ac{ICT}-Investitionen auf die Arbeitslosigkeit 
verschiedener Bildungsgruppen wurde bisher nicht ausreichend mit einer quantitativen, 
länderübergreifenden Panelanalyse untersucht.

Um diese Forschungslücken zu schließen, wird im empirischen Teil dieser Arbeit eine 
Paneldatenanalyse über \ac{OECD}-Länder durchgeführt. Dadurch sollen systematische 
Unterschiede in den Auswirkungen von Digitalisierung auf die Arbeitslosigkeit nach 
Bildungsniveaus erfasst werden.
