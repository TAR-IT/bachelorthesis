% TODO: Durchlesen! Zitation!

\section{Forschungsgegenstand}

Der Forschungsgegenstand dieser Arbeit umfasst die Untersuchung der Auswirkungen von 
Investitionen in \ac{ICT} auf den Arbeitsmarkt in \ac{OECD}-Ländern. Im Zentrum steht 
dabei die Frage, wie sich diese Investitionen auf die Beschäftigungslage, insbesondere die 
Arbeitslosenquote, in verschiedenen Bildungsgruppen auswirken. Dabei werden sowohl die 
nationalen Investitionen in digitale Technologien als auch die Auswirkungen auf 
Arbeitsmärkte und Beschäftigungsstrukturen betrachtet.

%%%%%%%%%%%%%%%%%%%%%%%%%%%%%%%%%%%%%
% Digitalisierung und Industrie 4.0 %
%%%%%%%%%%%%%%%%%%%%%%%%%%%%%%%%%%%%%

\subsection{Digitalisierung und Industrie 4.0}

Der Begriff der Digitalisierung beschreibt den zunehmenden Einsatz digitaler Technologien 
zur Automatisierung, Optimierung und Schaffung neuer Wertschöpfungspotenziale 
\parencite[S. 6]{brennen2016theinternational}. Im wirtschaftlichen Kontext wird dies mit 
der vierten industriellen Revolution (Industrie 4.0) in Verbindung gebracht, die durch die 
Integration von \ac{ICT}, \ac{AI}, Big Data, Cloud Computing, Cyber-Physischen Systemen 
sowie dem \ac{IoT} gekennzeichnet ist \parencite[S. 22]{kagermann2013recommendations}. 

Diese technologischen Fortschritte ermöglichen eine umfassende Automatisierung von 
Produktionsabläufen und eine Vernetzung einzelner Prozessschritte. Unternehmen profitieren 
durch höhere Effizienz, geringere Produktionskosten und eine flexiblere Fertigung. Ein 
zentraler Faktor ist die Möglichkeit, Daten in Echtzeit zu erfassen, zu analysieren und 
zur Prozessoptimierung zu nutzen, beispielsweise durch „vorausschauende Wartung“ 
(\textit{predictive maintenance}), wodurch Maschinenausfälle minimiert werden 
\parencite[S. 85]{bartodziej2016theconcept}. 

Der Wandel durch Industrie 4.0 hat weitreichende Implikationen für den Arbeitsmarkt. 
Einerseits entstehen neue hochqualifizierte Arbeitsplätze in den Bereichen Softwareentwicklung, 
Datenanalyse und Automatisierungstechnik. Andererseits führt die Digitalisierung in vielen 
Sektoren zur Verdrängung traditioneller Routinetätigkeiten, insbesondere in der industriellen 
Fertigung, im Transportwesen und in administrativen Berufen \parencite[S. 40]{frey2013thefuture}. 
Dadurch entstehen signifikante Unterschiede in der Betroffenheit verschiedener Bildungsgruppen 
vom technologischen Wandel.

Zudem verändert sich die Arbeitsorganisation durch flexible Arbeitszeitmodelle, Remote Work 
und Plattformarbeit \parencite[S. 112]{schwab2016thefourth}. Die digitale Vernetzung erfordert 
neue Kompetenzen und Anpassungsstrategien für Arbeitnehmer*innen, während gleichzeitig 
Fragen zu Datenschutz, IT-Sicherheit und algorithmengesteuerten Entscheidungsprozessen an 
Bedeutung gewinnen.

%%%%%%%%%%%%%%%%%%%%%
% ICT-Investitionen %
%%%%%%%%%%%%%%%%%%%%%

\subsection{ICT-Investitionen}

Investitionen in \ac{ICT} umfassen materielle und immaterielle Ressourcen, die zur 
Digitalisierung von Wirtschaft und Gesellschaft beitragen. Dazu zählen physische Infrastrukturen 
wie Glasfasernetze und Rechenzentren sowie immaterielle Investitionen in Software, Cloud-Dienste 
und digitale Plattformen \parencite[S. 15ff]{oecd2019measuring}. Sie sind ein wesentlicher 
Indikator für die digitale Transformation eines Landes und beeinflussen maßgeblich die 
Integration neuer Technologien in Produktions- und Dienstleistungsprozesse.

Der verstärkte Einsatz digitaler Technologien ermöglicht nicht nur Effizienzsteigerungen, 
sondern auch die Entwicklung neuer Geschäftsmodelle. Fortschritte in \ac{AI} und Big Data 
führen zur Automatisierung vieler Prozesse, wodurch Unternehmen flexibler auf 
Marktentwicklungen reagieren können \parencite[S. 15ff]{oecd2019measuring}. Gleichzeitig 
erlaubt der Einsatz von \ac{AI} eine effizientere Datennutzung, wodurch Entscheidungsprozesse 
optimiert werden.

\ac{ICT}-Investitionen sind zudem ein Schlüsselfaktor für wirtschaftliches Wachstum. Studien 
zeigen eine positive Korrelation zwischen Digitalisierung, Innovationsfähigkeit und der 
Wettbewerbsfähigkeit von Unternehmen \parencite[S. 22]{brynjolfsson2014thesecond}. Besonders 
in digital führenden Ländern wie den USA oder Deutschland sind diese Investitionen ein Treiber 
von Produktivitätssteigerungen. Gleichzeitig können sie dazu beitragen, strukturelle 
Ungleichheiten zwischen urbanen und ländlichen Regionen zu verringern.

Allerdings sind die Auswirkungen von \ac{ICT}-Investitionen auf den Arbeitsmarkt ambivalent. 
Während neue Arbeitsplätze in zukunftsorientierten Bereichen entstehen, führt die Automatisierung 
gleichzeitig zur Substitution von Routinetätigkeiten, insbesondere bei Berufen mit mittlerem 
Qualifikationsniveau \parencite[S. 40]{frey2013thefuture}. Dies verstärkt bestehende 
Arbeitsmarktungleichheiten.

%%%%%%%%%%%%%%%%%%%%%%%%%%%%%
% Arbeitsmarkt und Bildung %
%%%%%%%%%%%%%%%%%%%%%%%%%%%%%

\subsection{Arbeitsmarkt und Bildungsgruppen}

Der Arbeitsmarkt ist geprägt durch strukturelle Veränderungen infolge technologischer 
Innovationen. In dieser Arbeit liegt der Fokus auf der Arbeitslosigkeit nach Bildungsniveau, 
die üblicherweise in niedrig, mittel und hoch eingeteilt wird \parencite[S. 35–37]{frey2013thefuture}. 
Diese Differenzierung ermöglicht eine gezielte Analyse der Betroffenheit unterschiedlicher Gruppen.

Die Auswirkungen der Digitalisierung auf den Arbeitsmarkt werden häufig mit dem Konzept der 
Jobpolarisierung beschrieben. Hochqualifizierte Fachkräfte profitieren von der steigenden 
Nachfrage nach digitalen Kompetenzen, während Arbeitsplätze mit mittleren Qualifikationsanforderungen 
verstärkt unter Automatisierungsdruck geraten \parencite[S. 40]{autor2015whyare}. Geringqualifizierte 
sind besonders anfällig für Arbeitsplatzverdrängung, da ihre Tätigkeiten häufig leicht durch 
technologische Systeme ersetzt werden können \parencite[S. 10]{acemoglu2002technical}.

Frey und Osborne (2013) schätzen, dass rund 47~\% der Arbeitsplätze in den USA potenziell 
automatisierbar sind, wobei insbesondere Berufe mit niedriger Qualifikation betroffen sind 
\parencite[vgl.][S. 14–15]{frey2013thefuture}. In der Forschung wird daher diskutiert, ob 
technologischer Fortschritt die Kluft zwischen den Bildungsniveaus weiter vertieft und die 
Arbeitslosigkeit unter geringer Qualifizierten verstärkt \parencite[S. 2–4]{balsmeier2019isthis}.

Um den negativen Effekten entgegenzuwirken, werden gezielte bildungs- und arbeitsmarktpolitische 
Maßnahmen als notwendig erachtet. Besonders Weiterbildungsprogramme für digitale Kompetenzen 
sind zentral, um den Strukturwandel am Arbeitsmarkt abzufedern \parencite[S. 75]{brynjolfsson2014thesecond}. 
Länder mit einem gut ausgebauten Weiterbildungssystem können die negativen Folgen der 
Arbeitsmarktpolarisierung besser kompensieren.

Zusammenfassend hängt der Einfluss von \ac{ICT}-Investitionen auf den Arbeitsmarkt stark vom 
Bildungsniveau der Erwerbsbevölkerung ab. Während Hochqualifizierte profitieren, sehen sich 
niedrig und mittel qualifizierte Arbeitskräfte mit zunehmender Unsicherheit konfrontiert. Diese 
Entwicklung verdeutlicht die Notwendigkeit einer gezielten politischen Steuerung, um 
sozial-ökonomische Risiken zu minimieren.
