% TODO: Durchlesen!

\section{Daten und Methodik}

%%%%%%%%%%%%%%
% Datensätze %
%%%%%%%%%%%%%%

\subsection{Datensätze}

Die vorliegenden Daten, die auf den umfangreichen \ac{OECD}-Datensätzen beruhen, stellen eine 
solide Grundlage für die Untersuchung des Zusammenhangs zwischen Digitalisierung und 
Arbeitsmarkt dar. Insbesondere werden die Datensätze zu \ac{ICT}-Investitionen 
\parencite{oecd2022ict} sowie zu den Arbeitslosenquoten nach Bildungsniveau 
\parencite{oecd2022unemployment} verwendet. Diese Datensätze ermöglichen es, die Auswirkungen 
von Investitionen in \ac{ICT} auf die Arbeitslosigkeit in verschiedenen Bildungsgruppen auf 
internationaler Ebene zu analysieren, während gleichzeitig die geopolitischen Unterschiede 
berücksichtigt werden.

Die Daten zu \ac{ICT}-Investitionen messen die Bruttoanlageinvestitionen in digitale 
Infrastrukturen und Technologien \parencite{oecd2022ict}, während die Arbeitsmarktstatistiken 
detaillierte Informationen über die Arbeitslosenquoten in verschiedenen Bildungsgruppen 
bieten \parencite{oecd2022unemployment}. Diese Daten sind über mehrere Jahre hinweg verfügbar 
und ermöglichen somit eine aussagekräftige Paneldatenanalyse.

%%%%%%%%%%%%%%%%%%%%%%%
% Operationalisierung %
%%%%%%%%%%%%%%%%%%%%%%%

\subsection{Operationalisierung}

Um die Forschungsfrage - wie Investitionen in Informations- und Kommunikationstechnologien 
die Arbeitslosenquoten in unterschiedlichen Bildungsgruppen beeinflussen - fundiert zu 
beantworten, ist eine präzise und konsistente Operationalisierung der zentralen Konzepte und 
Variablen notwendig. Dies gewährleistet, dass die Untersuchung die beabsichtigten 
Zusammenhänge abbildet und die Daten sinnvoll ausgewertet werden können.

Die abhängige Variable dieser Untersuchung ist die \textit{Arbeitslosenquote} 
(UNEMPLOYMENT\_RATE\_PERCENT), differenziert nach dem Bildungsniveau der Bevölkerung. Die 
Arbeitslosenquote misst den Anteil der erwerbsfähigen Bevölkerung ohne Arbeit, wobei 
spezifische Bildungsniveaus berücksichtigt werden. Der \ac{OECD}-Datensatz unterteilt das 
Bildungsniveau in drei Hauptkategorien, die für diese Analyse relevant sind:

\begin{enumerate}
    \item \textbf{Niedriges Bildungsniveau}: Personen ohne 
    abgeschlossene Schulbildung oder mit einem maximalen Hauptschulabschluss.

    \item \textbf{Mittleres Bildungsniveau}: Personen mit 
    Sekundarschulabschluss oder einer abgeschlossenen Berufsausbildung.

    \item \textbf{Hohes Bildungsniveau}: Personen mit Hochschulabschluss, wie 
    einem Bachelor, Master oder Doktortitel.
\end{enumerate}

Arbeitslose sind nach dem \ac{OECD}-Datensatz Personen im erwerbsfähigen Alter, die keine 
Arbeit haben, die für eine Arbeit zur Verfügung stehen und die in den letzten vier Wochen 
konkrete Schritte unternommen haben, um eine Arbeit zu finden 
\parencite{oecd2022unemployment}. Diese Definition basiert auf internationalen Standards und 
wird durch Arbeitskräfteerhebungen erfasst. Der Indikator wird als Prozentsatz der 
Erwerbsbevölkerung gemessen und ist saisonbereinigt. Die Daten liegen länderspezifisch und 
zeitlich differenziert als Paneldaten über mehrere Jahre vor.

Die unabhängige Variable \textit{\ac{ICT}-Investitionen} (ICT\_INVEST\_SHARE\_GDP) umfasst 
Investitionen in digitale Infrastruktur, Software, Hardware und andere Technologien, die der 
Verbesserung betrieblicher Effizienz und Produktivität dienen \parencite{oecd2022ict}. Im 
\ac{OECD}-Datensatz werden \textit{\ac{ICT}-Investitionen} als Bruttoanlageinvestitionen in 
Informations- und Kommunikationsausrüstung sowie Computersoftware und -datenbanken gemessen, 
basierend auf den Definitionen des \ac{SNA08}. Für diese Arbeit werden 
\textit{\ac{ICT}-Investitionen} als Anteil des \ac{BIP} operationalisiert, wobei der Wert als 
Prozentwert angegeben wird. Die Daten liegen jährlich und länderspezifisch vor und sind somit 
für Paneldatenanalysen geeignet.

Um sicherzustellen, dass der Effekt der \textit{\ac{ICT}-Investitionen} auf die 
Arbeitslosenquote nicht durch andere Faktoren verzerrt wird, werden mehrere Kontrollvariablen 
integriert:

\begin{itemize}
    \item \textbf{\textit{\ac{BIP} pro Kopf} (GDP\_PER\_CAPITA):} Diese Variable misst den 
    wirtschaftlichen Wohlstand eines Landes in US-Dollar pro Jahr und kontrolliert den 
    Entwicklungsstand eines Landes, da wirtschaftlich wohlhabendere Länder tendenziell 
    niedrigere Arbeitslosenquoten aufweisen \parencite{oecd2022gdp}.

    \item \textbf{\textit{Gewerkschaftsdichte} (PERCENT\_EMPLOYEES\_TUD):} Der Anteil der in 
    Gewerkschaften organisierten Arbeitnehmer wird berücksichtigt, da Gewerkschaften eine 
    wichtige Rolle bei der Aushandlung von Arbeitsbedingungen spielen 
    \parencite{oecd2022tud}.
\end{itemize}

Diese Variablen stammen aus der \ac{OECD}-Datenbank und werden jährlich erhoben. Die 
Kombination dieser Daten ermöglicht es, länderspezifische Unterschiede in der 
Wirtschaftskraft, den regulatorischen Rahmenbedingungen und der Bildungsstruktur zu 
kontrollieren, um den direkten Einfluss der \textit{\ac{ICT}-Investitionen} präzise zu 
analysieren.

Die Analyse erfolgt über einen Zeitraum von 2005 bis 2022, wodurch langfristige Trends 
berücksichtigt werden können. Die Paneldatenstruktur ermöglicht es, zeitliche und 
länderspezifische Variationen einzubeziehen, was der Analyse zusätzliche Tiefe und 
Aussagekraft verleiht.

%%%%%%%%%%%%%%%%%%%%%%%
% Analytische Methode %
%%%%%%%%%%%%%%%%%%%%%%%

\subsection{Analytische Methode}

Die Analyse dieser Arbeit basiert auf einer Paneldatenanalyse, um die Auswirkungen 
von \textit{\ac{ICT}-Investitionen} auf die \textit{Arbeitslosenquote nach Bildungsniveau} 
zu untersuchen. Der methodische Ansatz kombiniert die Analyse zeitlicher Veränderungen und 
länderspezifischer Unterschiede und ermöglicht es, individuelle Heterogenität zwischen 
Ländern sowie dynamische Entwicklungen über die Zeit zu erfassen \parencite{wooldridge2010econometric}.

Zur Untersuchung der Effekte von \textit{\ac{ICT}-Investitionen} wurden zwei gängige 
Modelle der Paneldatenanalyse betrachtet: \ac{FE} und \ac{RE}. Diese Modelle unterscheiden sich 
in der Art und Weise, wie sie länderspezifische, zeitinvariante Effekte berücksichtigen:

\begin{itemize}
    \item \textbf{\ac{FE}-Modell:}  
    Kontrolliert für zeitinvariante länderspezifische Eigenschaften wie 
    geografische Lage, institutionelle Rahmenbedingungen oder kulturelle Unterschiede, 
    indem diese Effekte eliminiert werden. Es fokussiert sich ausschließlich auf die 
    Variation innerhalb eines Landes über die Zeit und ermöglicht eine genauere 
    Identifikation kausaler Effekte \parencite[S. 251–256]{wooldridge2010econometric}. 
    Ein Nachteil besteht darin, dass konstant bleibende länderspezifische Variablen 
    nicht geschätzt werden können.
    
    \item \textbf{\ac{RE}-Modell:}  
    Geht davon aus, dass länderspezifische Effekte zufällig verteilt und nicht mit den 
    unabhängigen Variablen korreliert sind. Es berücksichtigt sowohl die 
    Variation innerhalb als auch zwischen den Ländern \parencite[S. 17–20]{baltagi2021econometric}. 
    Falls die Annahme der Unkorreliertheit verletzt ist, können die Schätzungen verzerrt sein.
\end{itemize}

Für diese Analyse wurde das \ac{FE}-Modell gewählt, da es eine robustere Schätzung 
der kausalen Effekte von ICT-Investitionen auf die Arbeitslosigkeit ermöglicht. Dies ist 
besonders relevant, da die Untersuchung auf Veränderungen innerhalb eines Landes über die Zeit 
fokussiert und länderspezifische Eigenschaften nicht als erklärende Variablen modelliert werden.

Zudem werden über die Dummy-Variable \textit{Jahresfaktor} (YEAR\_FACTOR) zeitliche 
Faktoren wie Finanzkrisen und die Corona-Pandemie kontrolliert, um allgemeine makroökonomische 
Einflüsse auf die Ergebnisse zu minimieren. Hierbei wird das Jahr 2005 als Referenzkategorie 
gewählt. Neben den Haupteffekten der \textit{\ac{ICT}-Investitionen} und der Dummy-Variable 
werden Interaktionseffekte mit den \textit{Wohlfahrtsstaatentypen} modelliert, um 
institutionelle Rahmenbedingungen als moderierenden Faktor in der Beziehung zwischen 
Digitalisierung und Arbeitslosigkeit zu berücksichtigen \parencite{espingandersen1990thethree}. 
Die Wahl der Referenzkategorie „Anglo-Saxon“ erlaubt eine Interpretation der anderen 
\textit{Wohlfahrtsstaatentypen} relativ zu dieser Gruppe. 

Die Wahl der Referenzkategorie „Anglo-Saxon“ basiert auf mehreren Überlegungen:

\begin{itemize}
    \item \textbf{Repräsentativer Charakter:} Die angelsächsischen Länder (z. B. USA, 
    Großbritannien) stehen für ein liberales Wohlfahrtsmodell mit 
    marktorientierter Wirtschaftspolitik und vergleichsweise schwachen sozialen 
    Sicherungssystemen \parencite[S. 15–50]{brynjolfsson2015thesecond}.

    \item \textbf{Wissenschaftliche Relevanz:} Das anglo-sächsische Modell wird häufig als 
    Benchmark verwendet, da es eine Vorreiterrolle bei der Digitalisierung einnimmt 
    \parencite[Kap. 2]{brynjolfsson2015thesecond}.

    \item \textbf{Statistische Stabilität:} Die Referenzkategorie weist eine ausreichende 
    Fallzahl auf, um robuste Ergebnisse zu gewährleisten.
\end{itemize}

Die Kombination aus \ac{FE}-Modellen, Interaktionseffekten und 
zeitlichen Dummies bietet eine solide Grundlage für die Untersuchung der 
Auswirkungen von \textit{\ac{ICT}-Investitionen} auf die Arbeitslosigkeit. Die 
Paneldatenanalyse ermöglicht es, sowohl zeitliche Veränderungen als auch 
länderspezifische Unterschiede zu berücksichtigen, wodurch die komplexen Wechselwirkungen 
zwischen Digitalisierung, institutionellen Rahmenbedingungen und Bildungsniveaus präzise 
analysiert werden können \parencite[S. 12–15]{baltagi2021econometric}.
